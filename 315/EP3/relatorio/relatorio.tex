\documentclass[12pt]{article}
\usepackage[portuguese]{babel}               % acentos e tal
\usepackage[utf8]{inputenc}                  %
\usepackage{amsmath}						 % math

\usepackage{amsfonts}						 % math symbols (ex: R^n)
\usepackage{amssymb}

\usepackage{graphicx}						 % imagens
\usepackage{hyperref}                        % hyperlink
\usepackage[usenames, dvipsnames]{color}
% \usepackage[]{algorithm2e}
\usepackage{algpseudocode}
\usepackage{indentfirst}

% cores
\definecolor{title_color}{RGB}{60, 100, 200}
\definecolor{text_color}{RGB}{0, 0, 0}
\definecolor{section_color}{RGB}{25, 100, 75}
\definecolor{basic_dir}{RGB}{0, 255, 170}
\definecolor{svb}{RGB}{0, 50, 150}

% títulos
\usepackage{titlesec}
\titleformat*{\section}{\LARGE\bfseries}

\usepackage{helvet}
\renewcommand{\familydefault}{\sfdefault}

\title{Relatório EP3 MAC315}

\author{Bruno Sesso, Gustavo Estrela de Matos}

\bigskip
\date{\today}

\begin{document}
\begin{titlepage}
	
    \bigskip
    \bigskip
    \bigskip
    \bigskip
    \bigskip
    \bigskip

\begin{center}
	\begin{flushleft}
    	\textcolor{title_color} {
        	\fontsize{1cm}{1em}\selectfont {Implementação do Método Simplex\\}
    	}
    \end{flushleft}
  
	\bigskip
    \begin{flushleft}{
    \textcolor{text_color} {
    	\href{mailto:bsesso@gmail.com}{Bruno Sesso} 8536002\\    					      		           \href{mailto:estrela.gustavo.matos@gmail.com}{Gustavo Estrela de Matos} 8536051\\}
    }
    \end{flushleft}
    
	\bigskip
    \bigskip
    \bigskip
    \bigskip
    \bigskip
    \bigskip	\bigskip
    \bigskip
    \bigskip
    \bigskip
    \bigskip
    \bigskip	\bigskip
    \bigskip
    \bigskip
    \bigskip
    \bigskip
	\today
\end{center}
\end{titlepage}

\newpage
\section{Introdução}



\subsection{Apresentação do problema}
    Os problemas de Programação Linear (PL) são casos específicos de otimização combinatória em que a função objetivo e as restrições são ambos lineares. Portanto a função objetivo é da forma $c^{T}x$ e as restrições são da forma $a_{i}^{T}x \geq b_{i}$ ou $a_{i}^{T}x \leq b_{i}$, com $c, x, a_i \in \mathbb{R}^{n}$ e $b_{i} \in \mathbb{R}$.
    
    Multiplicando por $-1$ todas as restrições da forma $a_{i}^{T}x \geq b_i$, podemos escrever qualquer PL como:

    \begin{center}
    	\begin{tabular}{r l}
        	minimizar & $c^Tx$ \\
        
	        sujeito a & $Ax \leq b$, \\
        
        			  & $A \in R^{m \times n}$ e $b \in \mathbb{R}^m$.
		\end{tabular}
    \end{center}
	Também é possível mostrar que qualquer PL pode ser escrito na forma: 

	\begin{center}
    	\begin{tabular}{r l}
	  		minimizar & $c^Tx$ \\
        
	        sujeito a & $Ax = b$, \\
       				  & $x \geq 0$~\cite{315book}.
        \end{tabular}
    \end{center}
    Se for escrito dessa maneira, dizemos que o problema está no formato padrão. Adotaremos esse formato durante todo o trabalho.
    
    Se vale que $Ax^1 = b$ e $x^1 \geq 0$ dizemos que $x^1$ é um ponto viável. O conjunto $P = \{x| Ax = b, x \geq 0\}$ de todos os pontos viáveis é chamado conjunto viável.
    
    Uma solução ótima do problema é um ponto $x^1 \in P$ que minimiza \footnote[1]{Se o interesse for maximizar ${c}^{T}x$, podemos simplismente conseguir um problema equivalente em que o objetivo seja minimizar $-{c}^{T}x$.} a função objetivo $c$. Se $x^1$ existe, dizemos que o custo ótimo é $c^Tx$. Se $x^1$ não existe, ou não existem pontos viáveis ($P = \emptyset$), ou podemos diminuir o custo o quanto quisermos e dizemos que o custo ótimo é $- \infty$.
    

\subsection{Objetivos do trabalho}
	Neste trabalho, temos o objetivo de desenvolver, na linguagem Octave, o algoritmo simplex para resolver problemas de Programação Linear.


\section{Conceitos fundamentais}
	Antes de introduzirmos o funcionamento do nosso algoritmo, precisamos definir alguns conceitos que são fundamentais para garantir sua corretude.

	Seja o nosso problema de Programação Linear o seguinte:
    \begin{center}
    	\begin{tabular}{r l}
	  		minimizar & $c^Tx$ \\
        
        	sujeito a & $Ax = b$ \\
            & $x \geq 0$ \\
       	
        
        com & $c, x \in \mathbb{R}^n$, $A \in \mathbb{R}^{m \times n}$ e $b \in \mathbb{R}^{m}$.
        \end{tabular}
    \end{center}

	Além disso, vamos usar a notação $a_i$ para a i-ésima linha de A e $A_i$ para a i-ésima coluna de A.
    
    
\subsection{Restrições e degenerecência}
	Uma restrição $a^{T}_{i}x \geq b_{i}$ (ou $a^{T}_{i}x \leq b_{i}$), com $a_{i} \in \mathbb{R}^{n}$ e $b_i \in \mathbb{R}$, é uma \emph{restrição ativa} em um ponto $x^1 \in \mathbb{R}^n$ se $a^{T}_{i}x^1 = b_i$. Uma restrição de igualdade é sempre ativa. Um conjunto de restrições será dito LI se os vetores $a_{i}$ correspondentes forem LI.
    
    Diremos que $x^1$ uma solução viavel básica é \emph{degenerada} se existem mais de $n$ restrições ativas LI nesse ponto. Como as $m$ restrições de igualdade são sempre cumpridas, temos que as soluções básicas degeneradas possuem mais do que $n - m$ componentes nulas, enquanto que as não degeneradas possuem exatamente $n - m$.
    
    
\subsection{Soluções Viáveis Básicas}
	\label{subsec:svb}
    Dizemos que um ponto $x \in \mathbb{R}^{n}$ do conjunto viável $P$ é uma \emph{solução viável básica}, se existem $n$ restrições ativas em $x$ que são LI. Note que para problemas no formato padrão, existem sempre $m$ restrições ativas LI vindas de $Ax = b$, e as outras $n - m$ vem, necessariamente de $x \geq 0$. Portanto, uma solução viável básica possui ao menos $n - m$ componentes nulas.
    
    Se $x^1$ é uma solução básica não degenerada e seja $B(1), ..., B(m)$ os índices das componentes não nulas de $x$. A matriz $B = \begin{bmatrix} A_{B(1)}, ..., A_{B(m)}\end{bmatrix}$ é chamada \emph{matriz básica} associada a $x^1$.
    
    %teorema 2.8
    Se o conjunto $P$ tem uma solução viável básica, então ou o custo ótimo é $- \inf$ ou existe $x^1 \in P$ solução viável básica que é ótimo, ou seja, o custo de qualquer ponto do conjunto viável é maior ou igual do que o custo de $x^1$. Portanto, na solução de um PL com ao menos uma solução viável básica, podemos limitar a esses elementos a nossa busca por um ponto de custo ótimo~\cite{315book}.
    
    
\subsection{Direções básicas}
	Se $x^1$ é um solução viável básica de $P$, com índices básicos $B(1), ..., B(m)$. Dizemos que $d \in \mathbb{R}^n$, tal que $d_j = 1$, $Ad = 0$ $(A(x + \theta d) = b)$ e $d_i = 0$ para todo $i \notin \{B(1), ..., B(m)\}$, é a j-ésima direção básica partindo de $x^1$. Seja $d_B = \begin{bmatrix}d_{B(1)}, ..., d_{B(m)}\end{bmatrix}$, como $A(x + \theta d) = b$, temos que $d_B = -B^{-1}A_j$. Usaremos $u = -d_B = B^{-1}A_j$ por facilidade de notação, durante o trabalho.
 
 
	A figura \ref{fig:basic_dir} dá um exemplo de direções básicas, $di$ e $dj$ a partir de uma solução viável básica $x^1$. Note que o poliedro pode ou não limitar um $\theta$ tal que o ponto $y = x^1 + \theta*d$ ($d$ direção básica) seja viável, e como veremos em \ref{concepts:adj_svb} isso pode implicar em custo $-\infty$ se nessa direção o custo diminui.
    
\thicklines    
\begin{center}\label{fig:basic_dir}
    \begin{picture}(200, 200)
        \put(50, 50){\line(1, 0){100}}
        \put(50, 50){\line(-1, 1){50}}
        \put(35, 35){$x^2$}
        \put(50, 50){\circle*{5}}
        
        \put(-15, 85){\textcolor{svb}{$x^1$}}
        \put(0, 100){\line(1, 2){35}}
        % di
	    \put(0, 100){\textcolor{basic_dir}{\vector(1, 2){20}}}
		\put(5, 130){\textcolor{basic_dir}{$d_i$}}
        %dj
		\put(0, 100){\textcolor{basic_dir}{\vector(1, -1){25}}}
		\put(15, 65){\textcolor{basic_dir}{$d_j$}}
        \put(0, 100){\circle*{5}}
        
        \put(150, 50){\circle*{5}}
        \put(135, 35){$x^3$}
        
        \put(75, 100){$P$}
    \end{picture}
\end{center} 

\subsection{Custos reduzidos}\label{concepts:reduced costs}

    Seja $x^1$ uma solução viável básica, $B$ a matriz básica associada e $c_B = \begin{bmatrix} c_{B(1)}, ..., c_{B(m)}\end{bmatrix}$. Definimos, para cada $j \in \{1, ..., n\}$ o custo reduzido: 
    \begin{center}
    $\overline{c}_{j} = c_j - c_{B}^{T} B^{-1}A_j$.
    \end{center}
    
    %teorema 3.1
    Seja $x^1$ uma solução viável básica e $\overline{c}$ o vetor de custos reduzidos correspondente. Sabemos que se $\overline{c} \geq 0$, então $x^1$ é ótimo. Além disso, se $x^1$ for ótimo e não degenerado, então $\overline{c} \geq 0$~\cite{315book}. Portanto, se estivermos em uma solução viável básica e $\overline{c} \geq 0$, então estamos em um ponto ótimo.
    
    Note que ao escolhermos uma direção viável básica, o custo de um ponto $y = x^1 + \theta d_j$ é $c^{T}(x^1 + \theta d_j) = c^{T}x^1 + \theta (c_j - c^{T}_{B}B^{-1}A_j) = c^{T}x^1 + \theta \overline{c}$. Portanto, podemos dizer que o custo reduzido representa a variação do custo ao percorrer uma direção básica.

\subsection{Soluções Viáveis Básicas adjacentes}\label{concepts:adj_svb}
	Seja $x^1$ uma solução viável básica com índices básicos $B(1), ..., B(m)$. Uma solução viável básica é \emph{adjacente} a $x^1$ se compartilha $m-1$ índices com $x^1$. Para achar uma solução viável básica adjacente, vamos usar as direções básicas, pois elas forçam o crescimento de uma variável $j$ não-básica, mantendo $Ax = b$ e $x \geq 0$. Veremos que para um $\theta \geq 0$, o ponto $x^1 + \theta d_j$ é solução viável básica adjacente a $x^1$, com $d_j$ como foi definido em \ref{concepts:reduced costs}. 
    
    Vamos tomar $\theta = \min_{i = 1,...,m | u_i > 0}{\{x_{B(i)} / u_i\}}$ e ver que $x^2 = x^1 + \theta d_j$ é de fato uma solução viável básica adjacente a $x^1$. Caso todas as componentes de $u_i$ sejam menores ou igual a zero e o custo reduzido na direção $j$ menor do que zero teremos que o problema tem custo ótimo $- \infty$, como será explicado a seguir.
    
    Se $\theta$ definido acima não existe, temos que todas as componentes de $u_i$ são menores ou igual a zero ($d \geq 0$), logo qualquer ponto $x^2 = x^1 + \theta d$ é viável com $\theta \geq 0$, pois a restrição $Ax^2 = b$ é verificada (por construção), e $x^2_j = x^1_j + \theta \geq x^1_j \geq 0$, e para $i$ básico $x^2_j = x^1_j + \theta d_ j \geq x^1_j \geq 0$. Se ainda tivermos que o custo diminui nessa direção, poderemos diminuir o custo o quanto quisermos e a solução do problema será $- \infty$.
    
    Se $\theta \in \mathbb{R}$, como $d_i = 0$ $\forall{i \in \{B(1), ..., B(m)\}, i \neq j}$, temos que para essas mesmas componentes $x^2$ é nulo. Logo, temos $n - 1$ restrições ativas LI em $x^2$. Suponha que para $l \in \{1,..,m\}$ vale que $\theta =  x_{B(l)} / u_l$, então $x^2_{B(l)} = x^1_{B(l)} + (-x^1_B(l)/d_{B(l)}) * d_{B(l)} = 0$ (diremos que $B(l)$ sai da base), logo existem $n$ restrições ativas LI em $x^2$. Além disso, por construção, vale que $Ax = b$ e $x \geq 0$ para variáveis não básicas e para $x_B(l)$. Para $B(k)$ básico diferente de $B(l)$, temos que $x^2_B(k) \geq x^1_{B(k)} + (-x^1_B(k)/d_{B(k)}) * d_{B(k)} = 0$. Portanto $x^2$ é solução viável básica adjacente a $x^1$ e, como a base de $x^2$ é $\{B(1), ..., B(l - 1), j, B(l + 1), ..., B(m)\}$, $x^2$ é adjacente a $x^1$.
    
    
    
\section{O algoritmo}


\subsection{Ideia do algoritmo}
	A ultima seção apresenta ideias essenciais para a construção da fase 2 do algoritmo simplex. Dentre elas, as mais importantes são: podemos reduzir nosso espaço de busca as soluções viaveis básicas; se $\overline{c} \geq 0$ e estamos em uma solução viável básica, então esse ponto é ótimo.
    
    Portanto, utilizamos uma dinâmica que percorre as soluções viáveis básicas, com auxilio das direções básicas, sempre diminuindo a função custo, até que não seja mais possível sair de um ponto sem aumentar ou manter o custo, ou até encontrar uma direção que podemos diminuir o custo sem limitações.
    
    %o algoritmo pode parar se c >= 0 ou se u não tem componentes maiores que zero.
\subsection{Algoritmo}
\begin{algorithmic}
\Function{$simplex$}{$A$, $b$, $c$, $m$, $n$, $x$}
	\State calcula indices basicos ($Ib$) e não básicos ($In$)
    \State $B \gets  A_{Ib(i)}$, i = 1, ..., m
    \State $invB \gets B^{-1}$
   	\State $imin \gets 0$
	\If {$\nexists j$ t.q. $\overline{c_{j}} < 0$} 
        \State $\overline{c_{j}} \gets 0$
   	\Else
	\State $\overline{c_{j}} \gets c_{j} - c_{B}^{T}B^{-1}A_{j}$, algum $j \in In$ t.q. $\overline{c_{j}} < 0$
    \State $u \gets invB * A_{j}$  
    \EndIf 
    \While{$\overline{c_{j}} < 0$}
    	\If {$u_{l} < 0, l = 1, ... , m$}
        	\State \Return{$-1, d(u,j)$}
        \EndIf
    	\State $\theta \gets \min_{u_{l} >= 0} \{\frac{x_{Ib(l)}} {u_{l}}\}, l = 1, ... , m$
        \State $x \gets x + \theta * d(u, j)$
        \State $x_{Ib(l)}$ sai da base \Comment{Atualiza $In$}
        \State $x_{j}$ entra na base \Comment{Atualiza $Ib$}
        \State Atualiza $invB$
	    \State $\overline{c_{j}} \gets c_{j} - c_{B}^{T}B^{-1}A_{j}$, algum $j \in In$ t.q. $\overline{c_{j}} < 0$ 
        \State $u \gets invB * A_{j}$  
    \EndWhile
    \State \Return {$0, x$}
\EndFunction

\end{algorithmic}
    %colocar o exemplo ali em baixo, em resultados.%

%tenta seguir o esquema que eu to fazendo lá em baixo pra não ficar mto diferente


\section{Fase 1 do simplex}
Para iniciar a fase 2 do algoritmo simplex precisamos de uma solução viável básica $x^1$, a sua base associada e a inversa da matriz básica. Nesta seção explicaremos o funcionamento da fase 1 do método simplex, responsável por descobrir estes parâmetros.

Para descobrir esses parâmetros vamos criar um problema auxliar de programação linear:

\begin{center}
	\begin{tabular}{r l}
		minimizar & $\sum_{i = 1}^{m} y_i$ \\
		sujeito a & $
					[\begin{array}{c | c}
						A & I
					\end{array}]
					
					\bigg[\begin{array}{c}
						x \\
						\hline
						y
					\end{array}\bigg]
					
					\leq b$, \\
			com   &	$A \in R^{m \times n}$, $b, y \in \mathbb{R}^m$ e $I \in \mathbb{R}^{m \times m}$.
	\end{tabular}
\end{center}

\newpage
\begin{thebibliography}{9}
\bibitem{315book} Dimitris Bertsimas, John N. Tsitsiklis. Introduction to Linear Optimization. 1997.
\end{thebibliography}
\end{document}
